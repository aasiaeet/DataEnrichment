
\section{Estimation Algorithm}
\label{sec:opt}
We propose \emph{DAta SHarER} (\dc) a projected block gradient descent algorithm, Algorithm \ref{alg2}, where $\Pi_{\Omega_{f_g}}$ is the Euclidean projection onto the set $\Omega_{f_g}(d_g) = \{f_g(\bbeta) \leq d_g\}$ where $d_g = f_g(\bbeta_g^*)$ and is dropped to avoid cluttering. %In practice, $d_g$ can be determined by cross-validation.

\begin{algorithm}[t]
	\caption{  \dc }
	\label{alg2}
	\begin{algorithmic}[1]
		\STATE {\bfseries input:} $\X, \y$, learning rates $(\mu_0, \dots, \mu_G)$, initialization $\bbeta ^{(1)} = \0$
		\STATE {\bfseries output:} $\hbbe$
		\FOR{t = 1 \TO T}
		\FOR{g=1 \TO G}
		\STATE {\footnotesize $\bbeta _g^{(t+1)} = \Pi_{\Omega_{f_g}} \left(\bbeta _g^{(t)} + \mu_g \X_g^T \left(\y_g - \X_g \left(\bbeta _0^{(t)} + \bbeta _g^{(t)}\right) \right) \right)$}
		\ENDFOR
		\STATE {\footnotesize $\bbeta _0^{(t+1)} = \Pi_{\Omega_{f_0}} \left(\bbeta _0^{(t)} + \mu_0 \X_0^T \left(\y - \X_0 \bbeta _0^{(t)} -
		\begin{pmatrix}
		\X_1 \bbeta _1^{(t)}      \\
		\vdots 	 \\
		\X_G  \bbeta _G^{(t)}
		\end{pmatrix}\right)\right)$}
		\ENDFOR
	\end{algorithmic}
\end{algorithm}

To analysis convergence properties of \dc, we should upper bound the error of each iteration.
Let's $\ddelta^{(t)} = \bbeta^{(t)} - \bbeta^*$ be the error of  iteration $t$ of \dc, i.e., the distance from the true parameter (not the optimization minimum, $\hbbe$). We show that $\norm{\ddelta^{(t)}}{2}$ decreases exponentially fast in $t$ to the statistical error $\norm{\ddelta}{2} = \norm{\hbbe - \bbeta^*}{2}$. We first start with the required definitions for our analysis.

\begin{definition}
	\label{def:only}
	We define the following positive constants as functions of step sizes $\mu_g > 0$: %, where for simplification we assume $\X_0 = \oomega$ and $\oomega_0 = \oomega$:
%	\be
%	\nr
%	\rho_g(\mu_g) &=& \sup_{\u, \v \in \cB_g} \v^T \big(\I_g - \mu_g \X_g^T \X_g\big) \u, \quad g \in [G] \\ \nr
%	\eta_g(\mu_g) &=& \mu_g \sup_{\v \in \cB_g} \v^T \X_g^T \frac{\oomega_g}{\norm{\oomega_g}{2}}, \quad g \in [G] \\ \nr
%	\phi_g(\mu_g) &=& \mu_g \sup_{\v \in \cB_g, \u \in \cB_0} -\v^T \X_g^T \X_g \u, \quad g \in [G]_\setminus
%	\ee
	\be
	\nr
	\forall g \in [G_+]&:& \rho_g(\mu_g) = \sup_{\u, \v \in \cB_g} \v^T \big(\I_g - \mu_g \X_g^T \X_g\big) \u,
	\\ \nr
	&&\eta_g(\mu_g) = \mu_g \sup_{\v \in \cB_g} \v^T \X_g^T \frac{\w_g}{\norm{\w_g}{2}},
	\\ \nr
	\forall g \in [G]&:& \phi_g(\mu_g) = \mu_g \sup_{\v \in \cB_g, \u \in \cB_0} -\v^T \X_g^T \X_g \u,
	\ee
	where $\cB_g =  \cC_g \cap \ball$ and $\ball$ is the unit ball.%is the intersection of the error cone and the unit ball.% and $\oomega_0 := \oomega$.
\end{definition}
Below, we establish a deterministic bound on iteration errors  $\norm{\ddelta_g^{(t)}}{2}$ which depends on constants of Definition \ref{def:only} where to simplify the notation $\mu_g$ arguments are dropped. 
\begin{theorem}
	\label{theo:iter}
	The following deterministic bound for the error at iteration $t + 1$ of Algorithm \ref{alg2}, initialized by $\bbeta ^{(1)} = \0$, holds:
	\begin{align} %	\label{eq:singleiter}
	&\sum_{g=0}^{G} \sqrt{\frac{n_g}{n}} \norm{\ddelta_g^{(t+1)}}{2}
	\\ \nr 
	&\hspace*{1.2cm}\leq \rho^t \sum_{g=0}^{G}\sqrt{\frac{n_g}{n}}\norm{\bbeta ^*_g}{2}   + \frac{1 - \rho^t}{1 -  \rho}   \sum_{g=0}^{G} \sqrt{\frac{n_g}{n}} \eta_g \norm{\oomega_g}{2},
	\end{align}
	where {\small$\rho \triangleq \max\left(\rho_0 + \sum_{g=1}^{G} \sqrt{\frac{n_g}{n}} \phi_g, \max_{g \in [G]} \left[\rho_g + \sqrt{\frac{n}{n_g}}  \frac{\mu_0}{\mu_g} \phi_g \right]  \right)$}.
\end{theorem}

\begin{IEEEproof}
	First using the following lemma, we establish a recursive relation between errors of consecutive iterations which leads to a bound for the $t$th iteration. 
	
	\begin{lemma}
		\label{lem:recurse}
		The following recursive dependency holds between the error of $t+1$th and $t$th iterations of \dc{}:
		{\small\begin{align} 
		\nr 
		\norm{\ddelta_g^{(t+1)}}{2} &\leq   \rho_g(\mu_g)\norm{\ddelta_g^{(t)}}{2}   +  \xi_g(\mu_g) \norm{\oomega_g}{2} + \phi_g(\mu_g) \norm{\ddelta_0^{(t)}}{2} 
		\\ \nr 
		\norm{\ddelta_0^{(t+1)}}{2} &\leq   \rho_0(\mu_0) \norm{\ddelta_0^{(t)}}{2} + \xi_0(\mu_0) \norm{\oomega_0}{2} + \mu_0 \sum_{g=1}^{G}  \frac{\phi_g(\mu_g)}{\mu_g} \norm{\ddelta_g^{(t)}}{2}  
		\end{align} }
	\end{lemma}
	By recursively applying results of Lemma \ref{lem:recurse}, we get the following deterministic bound:% which depends on constants defined in Definition \ref{def:only}: 	
	{\small\begin{align}
	\nr 
	b_{t+1} &= \sum_{g=0}^{G} \sqrt{\frac{n_g}{n}} \norm{\ddelta_g^{(t+1)}}{2} 
	\leq  \left(\rho_0 + \sum_{g=1}^{G} \sqrt{\frac{n_g}{n}} \phi_g\right)  \norm{\ddelta_0^{(t)}}{2} 
	\\ \nr 
	&+ \sum_{g=1}^{G} \left(\sqrt{\frac{n_g}{n}} \rho_g + \mu_0 \frac{\phi_g}{\mu_g} \right) \norm{\ddelta_g^{(t)}}{2} + \sum_{g=0}^{G} \sqrt{\frac{n_g}{n}}  \xi_g \norm{\oomega_g}{2} 
%	\\ \label{eq:complicated}
	\\ \nr
	&\leq  \rho \sum_{g=0}^{G} \sqrt{\frac{n_g}{n}} \norm{\ddelta_g^{(t)}}{2} + \sum_{g=0}^{G} \sqrt{\frac{n_g}{n}}  \xi_g \norm{\oomega_g}{2}. 
	\end{align}	}
	%where {\small$\rho = \max\left(\rho_0 + \sum_{g=1}^{G} \sqrt{\frac{n_g}{n}} \phi_g, \max_{g \in [G]} \big[\rho_g + \sqrt{\frac{n}{n_g}}  \frac{\mu_0}{\mu_g} \phi_g \big]\right)$}. 
	
	We have:
	{\small
	\bea
	\nr  
	b_{t+1}
	&\leq  \rho b_{t} +  \sum_{g=0}^{G} \sqrt{\frac{n_g}{n}} \xi_g \norm{\oomega_g}{2} \\ \nr 
	&\leq \rho^2 b_{t-1}  + ( \rho + 1)  \sum_{g=0}^{G} \sqrt{\frac{n_g}{n}} \xi_g \norm{\oomega_g}{2} \\ \nr
	&\leq \rho^t b_1  + \left(\sum_{i = 0}^{t-1} \rho^i \right)   \sum_{g=0}^{G} \sqrt{\frac{n_g}{n}} \xi_g \norm{\oomega_g}{2} \\ \nr 
	&= \rho^t \sum_{g=0}^{G}\sqrt{\frac{n_g}{n}} \norm{\bbeta ^1_g  - \bbeta ^*_g}{2}  + \left(\sum_{i = 0}^{t-1} \rho^i \right)     \sum_{g=0}^{G} \sqrt{\frac{n_g}{n}} \xi_g \norm{\oomega_g}{2} \\ \label{eq:singleiter} 
	&\leq \rho^t \sum_{g=0}^{G}\sqrt{\frac{n_g}{n}} \norm{\bbeta ^*_g}{2}   + \frac{1 - \rho^t}{1 -  \rho}   \sum_{g=0}^{G} \sqrt{\frac{n_g}{n}} \xi_g \norm{\oomega_g}{2} 
	\eea}	
where the last inequality follows from $\bbeta ^1  = 0$.
\end{IEEEproof}



The RHS of \eqref{eq:singleiter} consists of two terms.
If we keep $\rho < 1$, the first term approaches zero fast, and the second term determines the bound. 
In the following, we show that for specific choices of step sizes $\mu_g$s we can keep $\rho < 1$ with high probability and the second term can be upper bounded using the analysis of Section \ref{sec:error}.
More specifically, the first term corresponds to the optimization error which shrinks in every iteration while the second term is of the same order of the upper bound of the statistical error characterized in Theorem \ref{theo:calcub}.

One way for having $\rho < 1$ is to keep all arguments of $\max(\cdots)$  defining $\rho$ strictly below $1$. %, i.e., $\rho_g < 1/\theta_f $.
To this end, we first establish high probability upper bound for $\rho_g$, $\eta_g$, and $\phi_g$ (in the Section \ref{twolems} of Supplement) and then show that with enough number of samples and proper step sizes $\mu_g$, $\rho$ can be kept strictly below one with high probability. %In Section~\ref{sec:expds}, we empirically illustrate such geometric convergence.
The high probability bounds for constants in Definition \ref{def:only} and the deterministic bound of Theorem \ref{theo:iter} leads to the following theorem which shows that for enough number of samples, of the same order as the statistical sample complexity of  Theorem \ref{theo:re}, we can keep $\rho$ below one and have geometric convergence.





%The following theorems uses Lemmas \ref{lemm:hpub} and \ref{lemm:mainlem} and shows for enough number of samples we can keep $\theta_f \rho$ below one.
%Bounds on $\rho_g\left(\frac{1}{n_g}\right)$ and $\eta_g\left(\frac{1}{n_g}\right)$ suggest that with enough number of samples, learning rate of $\mu_g = \frac{1}{n_g}$ leads to a linear rate of convergence to a constant times the statistical error bound in \ref{eq:singleiter}.
%The following theorem elaborates the result.
%Using Lemma \ref{lemm:hpub} and \ref{lemm:mainlem} bounds
%the following theorem shows that for a specific choice of step-sizes as $\mu_g = \frac{1}{n_g}$, $\rho_{\max} < 1$ with high probability and error of each iteration $\norm{\ddelta^{(t+1)}}{2}$ reaches to a scaled upper bound of statistical error $\norm{\ddelta}{2}$ exponentially fast.
\begin{theorem}
	\label{theo:step}		
	Let $\tau = \sqrt{\log(G+1)}/\zeta + \epsilon$ for $\epsilon, \zeta > 0$. For the step sizes of:
	\be
	\nr
	\mu_0 = \frac{\min_{g \in [G]} h_g(\tau)^{-2}}{4	n} ,
	\forall \in [G]: \mu_g =  \frac{h_g(\tau)^{-1}}{2\sqrt{n n_g}} 
	\ee
	where $h_g(\tau) = \left(1 + c_{0g} \frac{\omega(\cA_g) + \omega(\cA_0) + 2\tau}{\sqrt{n_g}}\right)$
	and sample complexities of $\forall g \in [G_+]: n_g \geq C_g (\omega(\cA_g) + \tau)^2$,
	with probability at least $ 1 - \sigma \exp(- \min(\nu \min_{g \in [G]} n_g - \log(G+1), \zeta \epsilon^2) )$ updates of Algorithm \ref{alg2} obey the following:	
	\begin{align}
	\nr
	&\sum_{g=0}^{G} \sqrt{\frac{n_g}{n}} \norm{\ddelta_g^{(t+1)}}{2}
	\leq r(\tau)^t \sum_{g=0}^{G} \sqrt{\frac{n_g}{n}} \norm{\bbeta^*_g}{2}   
%	+ \frac{(G+1) \sqrt{(2K^2 + 1)}}{\sqrt{n} (1 - r(\tau))}  \left(\zeta k \max_{g \in [G]} \omega(\cA_g) + \tau \right),
	\\ \nr 
	&\hspace*{1.75cm}+ \frac{C(G+1)\sqrt{(2k_w^2 + 1)k_x^2}}{\sqrt{n}(1 - r(\tau))} \left(\max_{g \in [G_+]} \omega(\cA_g) + \tau \right)
	\end{align}
	where $r(\tau) < 1$ is a constant depending on $\tau$ defined in \eqref{rtau} and $\upsilon, \zeta$, and $\sigma$ are constants.
	
\end{theorem}

\begin{corollary}
	\label{corr:show}
	For enough number of samples, iterations of DS algorithm with step sizes $\mu_0 = \Theta(\frac{1}{n})$ and $\mu_g =  \Theta(\frac{1}{\sqrt{n n_g}})$ geometrically converges to the following with high probability:
	{\small\beq
	\label{eq:scaled}
	\sum_{g=0}^{G} \sqrt{\frac{n_g}{n}} \norm{\ddelta_g^{\infty}}{2}
	\leq c \frac{\max_{g \in [G_+]} \omega(\cA_g) + \sqrt{\log (G+1)}/\zeta +  \theta}{\sqrt{n} (1 - r(\tau))}
	\eeq}
	where $c = C(G+1)\sqrt{(2k_w^2 + 1)k_x^2}$. 
\end{corollary}
	It is instructive to compare RHS of \eqref{eq:scaled} with that of \eqref{eq:general}: $\kappa_{\min}$ defined in Theorem \ref{theo:re} corresponds to $(1 - r(\tau))$ % defined in Theorem \ref{theo:step}
	and the extra $G+1$ factor corresponds to the sample condition number $\gamma = \max_{g \in [G] } \frac{n}{n_g}$.
	Therefore, Corollary \ref{corr:show} shows that with the number of samples in the order of sample complexity determined in Theorem \ref{theo:re} \dc{} converges to the statistical error bound determined in Theorem \ref{theo:calcub}.
	
\subsection{Proof Sketch of Theorem \ref{theo:step}}
\label{proofsketch}
	We want to determine $r(\tau) < 1$ such that $\rho < r(\tau)$ with high probability. Here, we provide a proof sketch using the below probabilistic bounds on constants of Definition \ref{def:only} while ignoring detailed computation of subsequent probabilities in finding $r(\tau)$. The full probabilistic proof is provided in Section \ref{twolems} of Supplement. 		
	First we need the following lemma to upper bound constants introduced in Definition \ref{def:only}: 
	
	\begin{lemma}
		\label{lemm:hpub}
		For $a_g \geq 1$ the following upper bounds hold:
		\begin{align}		
		\nr 
		&\rho_g\left(\frac{1}{a_g n_g}\right) \leq \frac{1}{2}  \left[\left(1 - \frac{1}{a_g} \right) + \sqrt{2} c_g\frac{2 \omega_g + \tau}{a_g \sqrt{n_g}} \right], %\quad 
		\\ \nr 
		&\hspace*{25pt}\text{with probability at least} 1 - 2\exp\left( -\gamma_g (\omega(\cA_g) + \tau)^2  \right).
		\\ \nr 
		&\eta_g\left(\frac{1}{a_g n_g}\right) \leq \frac{c_g k_x (\omega_g + \tau)}{a_g n_g}, %\quad 
		\\ \nr 
		&\hspace*{2.7cm}\text{with probability at least} 1 - \pi_g \exp\left( -\tau^2 \right).
		\\ \nr 
		&\phi_g\left(\frac{1}{a_g n_g}\right) \leq \frac{1}{a_g}  \left(1 + c_{0g}\frac{\omega_{0g} + 2\tau}{\sqrt{n_g}} \right), %\quad 
		\\ \nr 
		&\hspace*{25pt}\text{with probability at least} 1 - 2\exp\left( -\gamma_g (\omega(\cA_g) + \tau)^2  \right).
		\end{align} 
		where $\omega_g = \omega(\cA_g)$ and $\omega_{0g} = \omega(\cA_g) + \omega(\cA_0)$.
	\end{lemma}	
	To keep $\rho < 1$ in the deterministic bound of Theorem \ref{theo:iter} with the step sizes $\mu_g = \frac{1}{n_g a_g}$ we need to find the number of samples which satisfy the following conditions:
	
	\begin{itemize}
		\item Condition 1: $\rho_0\left(\mu_0\right) + \sum_{g=1}^{G} \sqrt{\frac{n_g}{n}} \phi_g\left(\mu_g\right) < 1$
		\item Condition 2: $\forall g \in [G]: \rho_g\left(\mu_g\right) + \sqrt{\frac{n}{n_g}} \frac{\mu_0}{\mu_g}\phi_g\left(\mu_g\right) < 1$
	\end{itemize}
	where according to the step sizes determine in the Theorem $a_0 \triangleq (4n\max_{g \in [G]}(1 + c_{0g}\frac{\omega_{0g} + 2\tau}{\sqrt{n_g}})^{2})^{-1}$ and $a_g \triangleq (2 \sqrt{n / n_g}(1 + c_{0g}\frac{\omega_{0g} + 2\tau}{\sqrt{n_g}}))^{-1}$.
	Condition 1 requires $\rho_0 + \sum_{g=1}^{G} \sqrt{\frac{n_g}{n}} \phi_g$ to be strictly below $1$ which is equivalent to: 
	\begin{align}	 
	\nr 
	&\rho_0\left(\mu_0\right) + \sum_{g=1}^{G} \sqrt{\frac{n_g}{n}} \phi_g\left(\mu_g\right) 
	\\ \nr 
	&\hspace*{45pt}\leq  \frac{1}{2}  \left[\left(1 - \frac{1}{a_0} \right) + \sqrt{2} c_0\frac{2 \omega_0 + \tau}{a_0 \sqrt{n}} \right] 
	\\ \nr 
	&\hspace*{45pt}+ \frac{1}{2} \sum_{g=1}^{G} \frac{2}{a_g}  \sqrt{\frac{n_g}{n}} \left(1 + c_{0g}\frac{\omega_{0g} + 2\tau}{\sqrt{n_g}} \right)
	\\ \nr 
	(i) &\hspace*{45pt}= \frac{1}{2}  \left[\left(1 - \frac{1}{a_0} \right) + \sqrt{2} c_0\frac{2 \omega_0 + \tau}{a_0 \sqrt{n}} \right] + \frac{1}{2} \sum_{g=1}^{G} \frac{n_g}{n} 
	\\ \nr 
	&\hspace*{45pt}=\frac{1}{2}  \left[\left(2 - \frac{1}{a_0} \right) + \sqrt{2} c_0\frac{2 \omega_0 + \tau}{a_0 \sqrt{n}} \right]  < 1
	\end{align}
	where we substitute $a_g$ value in step (i).
	So Condition 1 reduces to $n > 8 c_0^2 (\omega(\cA_0) + \tau)^2$. 
	
	Secondly in Condition 2, we want to bound all of $\rho_g + \mu_0 \sqrt{\frac{n}{n_g}} \frac{\phi_g}{\mu_g}$ terms for $\mu_g = \frac{1}{a_g n_g}$ by 1: {\small
	\begin{align}	
	\nr 
	&\rho_g\left(\mu_g\right) +  \sqrt{\frac{n}{n_g}} \frac{\mu_0}{\mu_g}\phi_g\left( \mu_g \right)
%	\\ \nr 
	= \rho_g\left(\frac{1}{n_g a_g}\right) +  \sqrt{\frac{n_g}{n}} \frac{a_g}{a_0}\phi_g\left(\frac{1}{n_g a_g}\right)
	\\ \nr
	&=	 \frac{1}{2} \left[\left(1 - \frac{1}{a_g} \right) + \sqrt{2} c_g\frac{2 \omega_g + \tau}{a_g \sqrt{n_g}} \right]  
	+  \frac{1}{a_0}  \sqrt{\frac{n_g}{n}} \left(1 + c_{0g}\frac{\omega_{0g} + 2\tau}{\sqrt{n_g}} \right)
	\\ \nr 
	&\leq 1
	\end{align} }	
	Condition 2 becomes: 	
	\be \nr 
	\sqrt{2} c_g\frac{2 \omega_g + \tau}{\sqrt{n_g}} 
	&\leq& 1 + a_g - \sqrt{\frac{n_g}{n}} \frac{2a_g}{a_0} \left(1 + c_{0g} \frac{\omega_{0g}+2\tau}{\sqrt{n_g}}\right)
	\\ \nr 
%	(a_g = 2\sqrt{\frac{n}{n_g}} (1 + c_{0g}\frac{\omega_{0g}+ \tau}{\sqrt{n_g}} ))
	(\text{Substitute } a_g)
	&=& 1 + a_g - \frac{4}{a_0} \left(1 + c_{0g} \frac{\omega_{0g}+2\tau}{\sqrt{n_g}}\right)^2
	\\ \nr 
%	(a_0 = 4	\max_{g \in [G]} (1 + c_{0g} \frac{\omega_{0g}+\tau}{\sqrt{n_g}})^2)
	(\text{Substitute } a_0)
	&\leq& 1 + a_g 
	\ee 
	So the sample complexity should be $\sqrt{n_g} > \frac{\sqrt{2} c_g(2 \omega_g + 2\tau)}{1+a_g}$ and since $a_g > 1$, the final per group sample complexity should be $n_g > 8c_g (\omega(\cA_g) + \tau)^2$.	\hfill {\qedsymbol}