\section{Restricted Eigenvalue Condition}
\label{sec:re}
The main assumption of \cref{theo:deter} is known as Restricted Eigenvalue (RE) condition in the literature of high-dimensional statistics \cite{banerjee14, nrwy12, raskutti10}:
$\inf_{\u \in \cH} \frac{1}{n} \norm{\X \u}{2}^2 \geq \kappa > 0.$
%\be
%\label{eq:recond}
%\inf_{\u \in \cH} \frac{1}{n} \norm{\X \u}{2}^2 \geq \kappa > 0.
%\ee
The RE condition posits that the minimum eigenvalues of the matrix $\X^T \X$ in directions restricted to $\cH$ is strictly positive.
In this section, we show that for the design matrix $\X$ defined in \cref{eq:x}, the RE condition holds with high probability under a suitable geometric condition we call {\em DAta SHaring Incoherence conditioN} (\ds) and for enough number of samples.
We precisely characterize total and per-group sample complexities required for successful parameter recovery.
For the analysis, similar to existing work \cite{guba16, mend15, trop15}, we assume the design matrix to be isotropic sub-Gaussian.\footnote{Extension to an-isotropic sub-Gaussian case is straightforward by techniques developed in \cite{banerjee14, ruzh13}.}
\begin{definition}
	\label{def:obs}
	We assume $\x_{gi}$ are i.i.d. random vectors from a non-degenerate zero-mean, isotropic sub-Gaussian distribution. In other words, $\ex [\x] = 0$, $\ex [\x^T \x] = \I_{p \times p}$, and $\normth{\x}{\psi_2} \leq k_x$.	
As a consequence, $\exists \alpha > 0$ such that $\forall \u \in \sphere$ we have $ \ex|\langle \x, \u \rangle| \geq \alpha$. Further, we assume noise $\w_{gi} $ are i.i.d.
zero-mean, unit-variance sub-Gaussian with $\normth{\w_{gi}}{\psi_2} \leq k_w$.
\end{definition}

\subsection{Geometric Condition for Recovery}
Unlike standard high-dimensional statistical estimation, for RE condition to be true, parameters of superposition models need to satisfy geometric conditions which limits the interaction of the error cones of parameters with each other to make sure that recovery is possible. In this section, we elaborate our sufficient geometric condition for recovery and compare it with state-of-the-art condition for recovery of superposition models. 

To intuitively illustrate the necessity of such a geometric condition, consider the simplest superposition model i.e., $\bbeta^*_0 + \bbeta^*_g$. Without any restriction on interactions of error cones, any estimates such that $\hbbe_0 + \hbbe_g = \bbeta^*_0 + \bbeta^*_g$ are valid ones. To avoid such trivial solutions two error cones need to satisfy $\ddelta_g \neq -\ddelta_0$. In general, the RE condition of individual superposition models can be established under the so-called Structural Coherence (SC) condition \cite{guba16, mctr13} which is the generalization of this idea for superposition of multiple parameters as $\sum_{g = 0}^{G} \bbeta^*_g$.
 
\begin{definition}[Structural Coherence (SC) \cite{guba16, mctr13}] \label{scc}
	Consider a superposition model of the form $y = \x^T \sum_{g = 0}^{G} \bbeta^*_g + w$. The SC condition requires that
%	\be 
%	\nr 
$	\forall \ddelta_g \in \cC_g, \exists \lambda \text{ s.t. }  \norm{\sum_{g = 0}^{G} \ddelta_g}{2} \geq  \lambda \sum_{g = 0}^{G}  \norm{\ddelta_g}{2}$,
%	\ee  
	and leads to the RE condition $\frac{1}{\sqrt{n}}\norm{\X \sum_{g = 1}^{G} \ddelta_g}{2} \geq \kappa \sum_{g=1}^{G} \norm{\ddelta_g}{2}$.
\end{definition}

\begin{remark}
	Note that the SC condition is satisfied if none of the individual error cones $\cC_g$ intersect with the inverted error cone $-\cC_0$ \cite{guba16, trop15}, i.e., $\forall g, \theta_g > 0$ in \cref{fig:sc} where 
	\be 
	\nr 
	\cos(\theta_g) = \sup_{\ddelta_0 \in \cC_0, \ddelta_g \in \cC_g} -\langle \ddelta_0/\norm{\ddelta_0}{2}, \ddelta_g/\norm{\ddelta_g}{2} \rangle.
	\ee
\end{remark}
Next, we introduce \ds, a considerably weaker geometric condition compared to SC which leads to recovery of all parameters in the data sharing model. 
\begin{definition}[DAta SHaring Incoherence conditioN (\ds)]  \label{incodef}
	There exists a non-empty set $\cI\subseteq [G]$ of groups where for some scalars $0 < \ratio\leq 1$ and $\lamin>0$ the following holds:
	\begin{enumerate}
		\item $\sum_{g\in \cI} n_g\geq \lceil \ratio n\rceil$.
		\item $\forall g \in \cI$, $\forall \ddelta_g \in \cC_g$, and $\ddelta_0\in\cC_0$: $\norm{\ddelta_g+\ddelta_0}{2}\geq \lamin (\norm{\ddelta_0}{2}+\norm{\ddelta_g}{2})$
	\end{enumerate}
	Observe that $0 < \lamin,\ratio\leq 1$ by definition.
\end{definition}
%\ab{why would $\bar{\rho}=0$ work? also, can $\cI$ be empty? pls update as needed} 

%\ab{add a remark -- so the reader can follow what we are saying}

%\ab{It will be great to have a Figure showing the difference between SC and \ds.}

\begin{remark}
	\ds\ is a refinement of SC for the specific problem of data sharing, i.e., system of coupled superposition model each with two components. \ds\ holds even if only one of the $\cC_g$s does not intersect with $-\cC_0$. More specifically, \ds\ holds if $\exists g, \theta_g > 0$ in \cref{fig:DASHIN} which allows $-\cC_0$ to intersect with an arbitrarily large fraction of the $\cC_g$ cones and as the number of intersections increases, our final error bound becomes looser.
\end{remark}

\subsection{Sample Complexity}
An alternative to our DS estimator \cref{eq:super} may be based on $G$ \emph{isolated} superposition model $\y_g = \X_g (\bbeta _0^* + \bbeta _g^*) + \w_g$ each with two components. Now, if SC holds for at least one of the superposition models, i.e., $\exists g, -\cC_0 \cap \cC_g = \{0 \}$, one can recover $\hbbe_0$ and plug it in to the remaining $G-1$ superposition estimators to estimate the corresponding $\hbbe_g$s. We call such an estimator, \emph{plugin superposition} estimator for which it seems that \ds\ has no advantage over SC. But the disadvantage of plugin superposition estimator is that it fails to utilize the true coupling structure in the data sharing model, where $\bbeta^*_0$ is involved in all groups. In fact, below we show that the plugin superposition estimator under SC condition leads to a pessimistic sample complexity for $\bbeta^*_0$ recovery.
\begin{prop}
	\label{prop:super}
	Assume observations distributed as defined in \cref{def:obs} and pair-wise SC conditions are satisfied.  Consider each superposition model \cref{eq:dirtymodel} in isolation; to recover the common parameter $\bbeta _0^*$ plugin superposition requires at least one group $i$ to have $n_i = O(\max(\omega^2(\cA_0), \omega^2(\cA_i)))$. 
	To recover the rest of individual parameters, it needs $\forall g \neq i: n_g = O(\omega^2(\cA_g))$ samples. 
\end{prop}
In other words, by separate analysis of superposition estimators at least one problem needs to have sufficient samples for recovering the common parameter $\bbeta_0$ and therefore the common parameter recovery does not benefit from the pooled $n$ samples.
But given the nature of coupling in the data sharing model, we hope to be able to get a better sample complexity specifically for the common parameter $\bbeta_0$.
Using \ds\ and the small ball method \cite{mend15}, a tool from empirical process theory in the following theorem, we get a better sample complexity required for satisfying the RE condition:
\begin{theorem}
	\label{theo:re}
	Let $\x_{gi}$s	be random vectors defined in \cref{def:obs}.
	Assume \ds\ condition of \cref{incodef} holds for error cones $\cC_g$s and $\rinc=\min\{1/2, \lamin\ratio/3\}$.
	Then, for all $\ddelta \in \cH$, when we have enough number of samples as $\forall g \in [G_+]: n_g \geq m_g = O(k_x^6 \alpha^{-6} \rinc^{-2} \omega(\cA_g)^2)$, with probability at least $1 - e^{-n \kappa_{\min}/4}$  we have:
	\be
	\nr
	\inf_{\ddelta \in \cH} \frac{1}{\sqrt{n}} \norm{\X \ddelta}{2} &\geq& \frac{\kappa_{\min}}{2}
	\ee
	where $\kappa_{\min} = \min_{g\in [G_+]} C \rinc \frac{\alpha^3}{k_x^2}  - \frac{2 c_g k_x \omega(\cA_g)}{\sqrt{n_g}}$. 
\end{theorem}

\begin{remark}
	Note that $\kappa = \frac{\kappa_{\min}^2}{4}$ is the lower bound of the RE condition of \cref{theo:deter}, i.e., $0 < \kappa \leq \inf_{\u \in \cH} \frac{1}{n} \norm{\X \u}{2}^2$ and is determined by the group with the worst individual RE condition. 
\end{remark}

\begin{example}
	{\bf ($l_1$-norm)} The Gaussian width of the spherical cap of a $p$-dimensional $s$-sparse vector is $\omega(\cA) = \Theta(\sqrt{s\log p})$ \cite{banerjee14, vershynin2018high}. Therefore, the number of samples per group and total required for satisfaction of the RE condition in the sparse DS estimator \cref{exm:sde} is $\forall g \in [G]: n_g \geq m_g = \Theta(s_g \log p)$. 
	\Cref{compare} compares sample complexities of the sparse DS estimator with three baselines: plugin superposition estimator of \cref{prop:super}, G Independent LASSO (GI-LASSO), and Jalali's Dirty Statistical Model (DSM) \cite{jrsr10}. Note that GI-LASSO does not recover the common parameter and DSM needs all groups have same number of samples. %\hfill {\color{header1} \qedsymbol}
\end{example}

\begin{table}[]
	\centering
	\begin{tabular}{l|c|c|c|c|}
		\cline{2-5}
		& \textbf{GI-LASSO} & \textbf{Dirty Stat. Model}                      & \textbf{Plugin Superposition}                                                                                                           & \textbf{Sparse DS} \\ \hline
		\multicolumn{1}{|l|}{\textbf{$m_g$}} & $s_{0g} \log p$   & $G \max_{g \in [G]} s_{0g} \log(p)$ & \begin{tabular}[c]{@{}c@{}}$\exists i \in [G]: \max(s_0, s_i) \log p$ \\ $\forall g \neq i: s_{g} \log p$\end{tabular} & $s_{g} \log p$     \\ \hline
	\end{tabular}
	\caption{Comparison of the order of per group number of samples (sample complexities) of various methods for recovering sparse DS parameters. Let $s_{0g} = |\text{support}(\bbeta^*_0 + \bbeta^*_g)|$ be the superimposed support where $s_0, s_g \leq \max(s_0, s_g) \leq s_{0g}$.}
	\label{compare}
\end{table}

%Sparse plus low rank example and sample complexity comparison table. 


\subsection{Proof of \cref{theo:re}}
Let's simplify the LHS of the RE condition:% \cref{eq:recond}: 
\be 
\nr 
\frac{1}{\sqrt{n}} \norm{\X \ddelta}{2} 
&=& \left(\frac{1}{n} \sum_{g=1}^{G} \sum_{i=1}^{n_g} |\langle \x_{gi}, \ddelta_0 + \ddelta_g \rangle|^2\right)^{\frac{1}{2}}
\\ \nr
%\text{(Lyapunov's Inequality)} 
\text{(Lyapunov's inequality)} &\geq& \frac{1}{n} \sum_{g=1}^{G} \sum_{i=1}^{n_g} |\langle \x_{gi}, \ddelta_0 + \ddelta_g \rangle| 
\\ \nr 
&\geq& \frac{1}{{n}} \sum_{g=1}^{G} \xi\norm{\ddelta_0+\ddelta_g}{2}  \sum_{i=1}^{n_g} \indic \left(|\langle \x_{gi}, \ddelta_0 + \ddelta_g \rangle| \geq \xi\norm{\ddelta_0+\ddelta_g}{2}\right)
\\ \nr 
&=& \frac{1}{{n}} \sum_{g=1}^{G} \xi_g  \sum_{i=1}^{n_g} \indic \left(|\langle \x_{gi}, \ddelta_{0g} \rangle| \geq \xi_g\right),
\ee 
where to avoid cluttering we denoted $\ddelta_{0g} = \ddelta_0 + \ddelta_g$ and $\xi_g = \xi \norm{\ddelta_{0g}}{2} > 0$.
%where $\ddelta_0 \in \cC_0$ and $\ddelta_g \in \cC_g$.
Now we add and subtract the corresponding per-group marginal tail function, $Q_{\xi_g}(\ddelta_{0g}) = \pr(|\langle \x, , \ddelta_{0g} \rangle| > \xi_g)$ and take $\inf$: 
\bea 
\nr
\inf_{\ddelta \in \cH} \frac{1}{\sqrt{n}} \norm{\X \ddelta}{2}
&\geq \inf_{\ddelta\in \cH}\sum_{g=1}^{G}  \frac{n_g}{n}  \xi_g  Q_{2 \xi_g}(\ddelta_{0g}) 
-	\sup_{\ddelta\in \cH} \frac{1}{n} \sum_{g=1}^{G}  \xi_g  \sum_{i=1}^{n_g} \left[Q_{2 \xi_g}(\ddelta_{0g})  - \indic (|\langle \x_{gi}, \ddelta_{0g} \rangle| \geq   \xi_g)  \right]
\\ \label{eq:long}
&= t_1(\X)-t_2(\X) 
\eea 
For the ease of exposition we consider the LHS of \cref{eq:long}  as the difference of two terms, i.e., $t_1(\X) - t_2(\X)$ and in the followings we lower bound the first term $t_1$ and upper bound the second term $t_2$. 

\subsubsection{Lower Bounding the First Term}
Our main result is the following lemma which uses the \ds\ condition of the \cref{incodef} and provides a lower bound for the first term $t_1(\X)$:
\begin{lemma}
	\label{lemm:shareInc} 
	Suppose \ds\ holds. Let $\rinc=\frac{\lamin\ratio}{3}$. For any $\ddelta \in \cH$, we have: 
	\beq 
	\label{eq:rhs}
	\sum_{g=1}^G\frac{n_g}{n}\xi_g Q_{2\xi_g}(\ddelta_{0g}) \geq \rinc\xi \frac{(\alpha - 2\xi)^2}{4ck_x^2}\sum_{g=0}^n \frac{n_g}{n}\norm{\ddelta_g}{2},
	\eeq 	
\end{lemma}
\cref{lemm:shareInc} implies that $t_1(\X)$
%$t_1(\X)=\inf_{\ddelta\in \cH} \sum_{g=1}^G\frac{n_G}{n}\xi_g Q_{2\xi_g}(\ddelta_{0g})$ 
is lower bounded by the same RHS bound of \cref{eq:rhs}.

\subsubsection{Upper Bounding the Second Term}
First we show $t_2(\X)$ satisfies the bounded difference property defined in Section 3.2. of \cite{boucheron13}, i.e., by changing each of $\x_{gi}$ the value of $t_2(\X)$ at most change by one. 
We rewrite $t_2$ as $t_2(\X) = \sup_{\ddelta \in \cH} g_\delta(\X)$ where $g_\delta\left(\X \right)$ is the argument of $\sup$ in \cref{eq:long}.
Now we denote the design matrix resulted from replacement of $k$th sample from $j$th group $\x_{jk}$ with another sample $\x'_{jk}$ by $\X'_{jk}$. Then our goal is to show $\forall j \in [G], k \in [n_j], \sup_{\X, \x_{jk}'} |t_2\left(\X \right)  - t_2(\X'_{jk} )|  \leq c_i$ for some constant $c_i$. 
Note that for bounded functions $f, g: \cX \rightarrow \reals$, we have $|\sup_{\cX} f - \sup_{\cX} g| \leq \sup_{\cX} |f - g|$. 
Therefore:
\be 
\nr 
\sup_{\X, \x_{jk}'} |t_2\left(\X \right)  - t_2\left(\X'_{jk} \right)|
&\leq& \sup_{\X, \x_{jk}'} \sup_{\ddelta \in \cH} \big|g\left(\X \right) - g\left(\X'_{jk} \right) \big|
\\ \nr 
&\leq& \sup_{\x_{jk},  \x_{jk}'} \sup_{\ddelta \in \cH} \frac{\xi_j}{n} \left|\indic (|\langle \x_{jk}', \ddelta_{0j}\rangle| \geq   \xi_j)  - \indic (|\langle \x_{jk}, \ddelta_{0j} \rangle| \geq   \xi_j) \right| 
\\ \nr 
&\leq& \sup_{j} \sup_{\ddelta \in \cH} \frac{\xi_j}{n} 
= \frac{\xi}{n} \sup_{j} \sup_{\ddelta \in \cH} {\norm{\ddelta_0 + \ddelta_j}{2}}
\\ \nr 
&\leq& \frac{\xi}{n} \sup_{j} \sup_{\ddelta \in \cH} \norm{\ddelta_0}{2} + \norm{\ddelta_j}{2}
\\ \nr 
(\ddelta \in \cH) &\leq& \xi \left(\frac{1}{n} + \frac{1}{{n_j}}\right) 
%\\ \nr 
\leq  \frac{2\xi}{n}
\ee 
Note that for $\ddelta \in \cH$ we have $\norm{\ddelta_0}{2} + \frac{n_g}{n}\norm{\ddelta_g}{2} \leq 1$ which results in $\norm{\ddelta_0}{2} \leq 1$ and $\norm{\ddelta_g}{2} \leq \frac{n}{n_g}$ which justifies the last inequality. 
Now, we can invoke the bounded difference inequality from Theorem 6.2 of \cite{boucheron13} which says that with probability at least $1 - e^{-\tau^2/2}$ we have: $t_2(\X) \leq \ex t_2(\X) + \frac{\tau}{\sqrt{n}}$. 
Having this concentration bound, it is enough to bound the expectation of $t_2(\X)$ using the following lemma:
\begin{lemma}
	\label{lemm:secTerm}
	For the random vector $\x$ of \cref{def:obs}, we have the following bound:
	\be 
	\nr 
	\frac{2}{n} \ex \sup_{\ddelta \in \cH} \sum_{g=1}^{G} \xi_g \sum_{i=1}^{n_g} \left[Q_{2 \xi_g}(\ddelta_{0g})  - \indic (|\langle \x_{gi}, \ddelta_{0g} \rangle| \geq \xi_g )  \right]
	\leq \frac{2}{\sqrt{n}} \sum_{g=0}^{G}  \sqrt{\frac{n_g}{n}} c_g k \omega(\cA_g) \norm{\ddelta_{g}}{2}
	\ee 
\end{lemma}

\subsubsection{Continuing the Proof of \cref{theo:re}}
Define $q \triangleq \frac{(\alpha - 2\xi)^2}{4ck^2}$. Putting back bounds of $t_1(\X)$ and $t_2(\X)$ together from \cref{lemm:shareInc,lemm:secTerm}, with probability at least $1 - e^{-\frac{\tau^2}{2}}$ we have:

\be
\nr 
\inf_{\ddelta \in \cH} \frac{1}{\sqrt{n}} \norm{\X \ddelta}{2}
%&\geq& \sum_{g=0}^{G}  \frac{n_g}{n} \rinc \xi \norm{\ddelta_g}{2} \frac{(\alpha - 2\xi)^2}{4ck^2}
%- \frac{2}{\sqrt{n}} \sum_{g=0}^{G}  \sqrt{\frac{n_g}{n}} c_g k \omega(\cA_g) \norm{\ddelta_{g}}{2} - \frac{\tau }{\sqrt{n}}
%\\ \nr
&\leq&\sum_{g=0}^{G}  \frac{n_g}{n} \rinc \xi \norm{\ddelta_g}{2} q
- \frac{2}{\sqrt{n}} \sum_{g=0}^{G}  \sqrt{\frac{n_g}{n}} k_x c_g \omega(\cA_g) \norm{\ddelta_{g}}{2} - \frac{\tau }{\sqrt{n}}
\\ \nr
&=&n^{-1}\sum_{g=0}^{G} n_g \norm{\ddelta_{g}}{2} ( \rinc \xi  q-2 c_g k_x \frac{\omega(\cA_g)}{\sqrt{n_g}})-\frac{\tau}{\sqrt{n}}
\\ \nr
(\kappa_g = \rinc \xi q  - \frac{2 c_g k_x \omega(\cA_g)}{\sqrt{n_g}}) &=& \sum_{g=0}^{G} \frac{n_g}{n} \norm{\ddelta_g}{2} \kappa_g  - \frac{\tau}{\sqrt{n}}
\\ \nr
&\geq& \kappa_{\min}\sum_{g=0}^{G} \frac{n_g}{n} \norm{\ddelta_g}{2}  - \frac{\tau}{\sqrt{n}}
\\ \nr
%(n_g \geq 1) &\geq& \kappa_{\min} \sqrt{\frac{1}{n}} \sum_{g=0}^{G} \sqrt{\frac{n_g}{n}} \norm{\ddelta_g}{2} - \frac{\tau}{\sqrt{n}}
%\\ \nr
(\ddelta \in \cH) &=& \kappa_{\min}  - \frac{\tau}{\sqrt{n}} %= \kappa 
\ee
where $\kappa_{\min} = \min_{g\in [G]} \kappa_g$. To conclude the proof, take $\tau = \sqrt{n} \kappa_{\min}/2$. 

To satisfy the RE condition all $\kappa_g$s should be bounded away from zero.
To this end we need the following sample complexities $\forall g \in [G_+]: \left(\frac{2 c_g k }{\rinc \xi q}\right)^2 \omega(\cA_g)^2 \leq n_g $ where by taking $\xi = \frac{\alpha}{6}$ simplifies to: $\forall g \in [G_+]: O\left(k^6 \rinc^{-2} \alpha^{-6} \omega(\cA_g)^2\right) \leq n_g$ \hfill {\color{header1} \qedsymbol}

%\subsection{Proof of Theorem \cref{theo:ub} with $\frac{n_g}{n}$}
%\begin{proof}	
%	%First we provide an upper bound for the expectation of $2\oomega^T \X\ddelta$ and in the next step we should concentration around the mean by large deviation bound. 
%From now on, to avoid cluttering the notation assume $\oomega = \oomega_0$.
%We massage the equation as follows:
%\be 
%\nr 
%\oomega^T \X\ddelta &=& \sum_{g=0}^{G} \langle \X_g^T \oomega_g,  \ddelta_g \rangle 
%= \sum_{g=0}^{G} {\frac{n_g}{n}} \norm{\ddelta_g}{2} \langle \X_g^T \frac{\oomega_g}{\norm{\oomega_g}{2}}, \frac{\ddelta_g}{\norm{\ddelta_g}{2}} \rangle {\frac{n}{n_g}} \norm{\oomega_g}{2} \nr
%%(\forall g: \u_g \in \cC_g \cap \sphere) &=& \norm{\ddelta_0}{2} \langle \X_0^T \frac{\oomega}{\norm{\oomega}{2}}, \u_0 \rangle \norm{\oomega}{2} + \sum_{g=1}^{G} \norm{\ddelta_g}{2} \langle \X_g^T \frac{\oomega_g}{\norm{\oomega_g}{2}}, \u_g \rangle \norm{\oomega_g}{2} \\ \nr
%\ee	
%
%
%Assume $b_g = \langle \X_g^T \frac{\oomega_g}{\norm{\oomega_g}{2}}, \frac{\ddelta_g}{\norm{\ddelta_g}{2}}  \rangle {\frac{n}{n_g}} \norm{\oomega_g}{2}$ and $a_g = {\frac{n_g}{n}} \norm{\ddelta_g}{2}$. 
%Then the above term is the inner product of two vectors $\a = (a_0, \dots, a_G)$ and $\b = (b_0, \dots, b_G)$ for which we have:
%\be 
%\nr 
%%\sup_{\a \in \cH_l} \a^T \b  &\leq& \sup_{\a \in \cH_0} \a^T \b 
%%\\ \nr
%\sup_{\a \in \cH} \a^T \b 
%&=&\sup_{\norm{\a}{1} = 1} \a^T \b \\ \nr
%\text{(definition of the dual norm)} &\leq& \norm{\b}{\infty} \\ \nr 
%&=& \max_{g \in [G]} b_g \nr  
%\ee 
%%where $\cH_0 \supset \cH_l$ defines as:
%%\be 
%%\nr 
%%\cH_0 &=& \left\{ \ddelta = (\ddelta_0^{(t)} , \dots, \ddelta_g^{(t)})^T \Big| \forall g \in [G]: \ddelta_g \in \cC_g, \sum_{g=0}^{G} {\frac{n_g}{n}} \norm{\ddelta_g}{2} \leq 1 \right\}, %\label{setH}
%%\ee 
%Now we can go back to the original form:
%\be 
%%\label{eq:maxex}
%\sup_{\ddelta \in \cH}\oomega^T \X\ddelta
%&\leq& \max_{g \in [G]} \langle \X_g^T \frac{\oomega_g}{\norm{\oomega_g}{2}}, \frac{\ddelta_g}{\norm{\ddelta_g}{2}}  \rangle {\frac{n}{n_g}} \norm{\oomega_g}{2} \\ 
%\nr 
%&\leq& \max_{g \in [G]} {\frac{n}{n_g}} \norm{\oomega_g}{2} \sup_{\u_g \in \cC_g \cap \sphere} \langle \X_g^T \frac{\oomega_g}{\norm{\oomega_g}{2}}, \u_g \rangle 
%\ee 
%
%To avoid cluttering we name $h_g(\oomega_g, \X_g) =  \norm{\oomega_g}{2} \sup_{\u_g \in \cA_g} \langle \X_g^T \frac{\oomega_g}{\norm{\oomega_g}{2}}, \u_g \rangle $ and $e_g(\tau) =  \sqrt{(2K^2 + 1)n_g} \left(\upsilon_g C_gk \omega(\cA_g) + \epsilon_g \sqrt{\log G} + \tau \right)$.
%Then from \cref{eq:maxex}, we have: {\color{red} This step won't work.
%\be
%\nr  
%\pr \left(\frac{2}{n} \sup_{\ddelta \in \cH} \oomega^T \X\ddelta >  \frac{2}{n} \max_{g \in [G]} \sqrt{\frac{n}{n_g}} e_g(\tau) \right) 
%&\leq& \pr \left(\frac{2}{n} \sup_{\ddelta \in \cH} \oomega^T \X\ddelta >  \frac{2}{n} \max_{g \in [G]} {\frac{n}{n_g}} e_g(\tau) \right) 
%\\ \nr 
%&\leq& \pr \left(\frac{2}{n} \max_{g \in [G]} {\frac{n}{n_g}} h_g(\oomega_g, \X_g) > \frac{2}{n} \max_{g \in [G]} {\frac{n}{n_g}} e_g(\tau) \right) 
%\ee }
%To simplify the notation, we drop arguments of $h_g$ for now. 
%From the union bound we have:
%\be
%\nr 
%\pr \left(\frac{2}{n} \max_{g \in [G]} {\frac{n}{n_g}} h_g >  \frac{2}{n} \max_{g \in [G]} {\frac{n}{n_g}} e_g(\tau) \right)  
%&\leq& \sum_{g=0}^{G} \pr \left(h_g >  \max_{g \in [G]}  e_g(\tau) \right)  \\ 
%\nr 
%&\leq& \sum_{g=0}^{G} \pr \left( h_g >  e_g(\tau) \right)  \\ 
%\nr 	
%&\leq& G \max_{g \in [G]} \pr \left(h_g > e_g(\tau) \right) \\ 
%\nr 
%&\leq& \sigma \exp\left(-\min_{g \in [G]}\left[\nu_g  n_g - \log G, \frac{\tau^2}{\eta_g^2 k^2}\right]\right) 
%\ee 
%where $\sigma = \max_{g \in [G]} \sigma_g$. 	 
%\end{proof} 


