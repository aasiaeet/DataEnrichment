\section{The Data Enrichment Estimator}
\label{sec:esti}
%Given $G$ group and $n_g$ samples in each one as $\{ \{\x_{gi}, y_{gi} \}_{i=1}^{n_g} \}_{g = 1}^G$, we can form the per group design matrix $\X_g \in \reals^{n_g \times p}$ and output vector $\y_g \in \reals ^{n_g}$.
%The total number of samples is  $n = \sum_{g = 1}^{G} n_g$.
%The data enriched model takes the following vector form:
%\be
%\label{eq:dirtymodel}
%\y_g = \X_g (\bbeta _0^* + \bbeta _g^*) + \oomega_g,  \quad \forall g \in [G]_\setminus
%\ee
%where each row of $\X_g$ is $\x_{gi}^T$ and $\oomega_g^T = (\omega_{g1}, \dots, \omega_{gn_g})$ is the noise vector. %consists of i.i.d. centered unit-variance sub-Gaussian elements with $\normth{\omega_{gi}}{\psi_2} \leq K$.
%Note that,
%The common parameter among all groups is $\bbeta _0^*$ and the individual parameter of the group $g$ is $\bbeta _g^*$.
%Beside a deterministic error bound presented in Section \ref{sec:deter} our other results are probabilistic.
%We focus on independent isotropic sub-Gaussian random vectors $\x_{gi}$ where $\normth{\x_{gi}}{\psi_2} \leq k$ and $\ex \x_{gi}^T \x_{gi}  = \I_{p \times p}$.

%\subsection{The Estimator}
A compact form of our proposed DE estimator \eqref{eq:super} is:% optimization problem as: %:
\be
\label{eq:compact}
\hbbe \in \argmin_{\bbeta } \frac{1}{n} \norm{\y - \X \bbeta }{2}^2, \quad \text{s.t. } \forall g \in [G] \cup \{0\}:f_g(\bbeta_g ) \leq f_g(\bbeta^*_g),
\ee
where $\y  = (\y^T_1, \dots \y^T_G)^T \in \reals^n$,  $\bbeta  = ({\bbeta _0}^T, \dots, {\bbeta _G}^T)^T \in \reals^{(G+1)p}$ and
\be
\label{eq:x}
\X =
\begin{pmatrix}
	\X_1     & \X_1      & 0      	   & \cdots & 0 \\
	\X_2     & 0       	 & \X_2        & \cdots & 0 \\
	\vdots 	 & \vdots  	 & \ddots 	   & \cdots & \vdots  \\
	\X_G     & 0       	 & \cdots 	   & \cdots & \X_G
\end{pmatrix}
\in \reals^{n \times (G+1)p}~.
\ee
%For simplicity, during steps of the analysis we denote $\X = [\X_0 \enskip \D]$ which is the concatenation of $\X_0 \in \reals^{n \times p}$ that represents the \emph{whole} design matrix consisting of all data points as rows and $\D \in \reals^{n \times pG}$ which is the \emph{diagonal} part of the $\X$ where all $\X_g$s are on the diagonal.
%Following this convention, we refer to total number of samples $n$ as $n_0$ in our analysis.

%\ab{Give at least 2 specific examples, e.g., a sparse+sparse for vector estimation, and a low-rank + sparse for matrix estimation. And we want to carry these examples through
%the technical results, e.g., see the Chen-Banerjee NIPS 2015 paper.}

\begin{example}
{\bf ($L_1$-norm)} When all parameters $\bbeta_g$s are $s_g$-sparse, i.e.,$|\text{supp}(\bbeta_{g}^*)| = s_g$ by using $l_1$-norm as the sparsity inducing function, our DE estimator of \eqref{eq:compact} becomes:
\be 
\label{sde}
\hbbe \in \argmin_{\bbeta } \frac{1}{n} \norm{\y - \X \bbeta }{2}^2, \quad \text{s.t. }  \forall g \in [G] \cup \{0\}: \norm{\bbeta_g}{1} \leq \norm{\bbeta^*_g}{1}.
\ee 
We call \eqref{sde} \emph{sparse DE } estimator and use it as the running example throughout the paper to illustrate outcomes of our analysis.
\end{example}
Consider the group-wise estimation error $\ddelta_g = \hbbe_g - \bbeta^*_g$.
Since $\hbbe_g = \bbeta ^*_g + \ddelta_g$ is a feasible point of \eqref{eq:compact}, the error vector $\ddelta_g$ will belong to the following restricted error set:% which is the set of all descent directions at $\bbeta _g^*$ on $f_g(\cdot)$ :
\be
%\nr
\cE_g = \left\{\ddelta_g | f_g(\bbeta _g^* + \ddelta_g) \leq f_g(\bbeta _g^*)\right\}, \quad g \in [G] \cup \{0\}.
\ee
We denote the cone of the error set as $\cC_g \triangleq \text{Cone}(\cE_g)$ and the spherical cap corresponding to it as $\cA_g \triangleq \cC_g \cap \sphere$.
Consider the set $\cC = \{ \ddelta = (\ddelta_0^T, \dots, \ddelta_G^T)^T \Big| \ddelta_g \in \cC_g \}$, following two subsets of $\cC$ play key roles in our analysis:
%\be
%%\nr
%\cH  &\triangleq&  \Big\{ \ddelta \in \cC \big| \sum_{g=0}^{G} {\frac{n_g}{n}} \norm{\ddelta_g}{2} = 1 \Big\}~, %\label{setH}
%\\ %\nr
%\bcH &\triangleq&  \Big\{ \ddelta \in \cC \big| \sum_{g=0}^{G} \sqrt{\frac{n_g}{n}} \norm{\ddelta_g}{2} = 1 \Big\}~. %\label{setH}
%\ee
\be
%\nr
\label{setH}
\cH  \triangleq  \Big\{ \ddelta \in \cC \big| \sum_{g=0}^{G} {\frac{n_g}{n}} \norm{\ddelta_g}{2} = 1 \Big\}~, 
\quad 
\bcH \triangleq  \Big\{ \ddelta \in \cC \big| \sum_{g=0}^{G} \sqrt{\frac{n_g}{n}} \norm{\ddelta_g}{2} = 1 \Big\}~. %\label{setH}
\ee
Starting from the optimality of $\hbbe = \bbeta ^* + \ddelta$ as $\frac{1}{n}\norm{\y - \X \hbbe}{2}^2 \leq \frac{1}{n} \norm{\y - \X \bbeta ^*}{2}^2$, we have: $\frac{1}{n}\norm{\X \ddelta}{2}^2 \leq \frac{1}{n}2\w^T \X\ddelta$ where $\w = [\w_1^T, \dots, \w_G^T]^T \in \reals^n$ is the vector of all noises.
Using this basic inequality, we can establish the following deterministic error bound.
\begin{theorem}
	\label{theo:deter}
	For the DE estimator \eqref{eq:compact}, assume there exist $0 < \kappa \leq \inf_{\u \in \cH} \frac{1}{n} \norm{\X \u}{2}^2$. Then, for the sample condition number $\gamma = \max_{g \in [G]} \frac{n}{n_g}$, the following deterministic upper bounds holds:
	\be
	\nr
	\sum_{g=0}^{G} \sqrt{\frac{n_g}{n}} \norm{\ddelta_g}{2} \leq \frac{2{\gamma}\sup_{\u \in \bcH}\oomega^T \X \u}{n\kappa}~. %\nr	
	\ee
\end{theorem}
\begin{proof}
	We lower bound the LHS and upper bound the RHS of the optimality inequality $\frac{1}{n}\norm{\X \ddelta}{2}^2 \leq \frac{1}{n}2\w^T \X\ddelta$ using the definition of the sets $\cH$ and $\bcH$ respectively. 
	Starting with the lower bound using the definition of set $\cH$ \eqref{setH} we have:
	\be 
	\label{eq:tre} 
	\frac{1}{n}\norm{\X \ddelta}{2}^2 &\geq& \frac{1}{n} \inf_{\u \in \cH} \norm{\X \u}{2}^2  \left(\sum_{g=0}^{G} {\frac{n_g}{n}} \norm{\ddelta_g}{2} \right)^2 \\ \nr
	&\geq& \kappa  \left(\sum_{g=0}^{G} {\frac{n_g}{n}} \norm{\ddelta_g}{2} \right)^2  \nr
	\\ \nr 
	&\geq& \kappa  \left(\min_{g \in [G] } \frac{n_g}{n}\right) \left(\sum_{g=0}^{G} \sqrt{\frac{n_g}{n}} \norm{\ddelta_g}{2} \right)^2  
	\ee 
	where $0 < \kappa \leq \frac{1}{n}  \inf_{\u \in \cH} \norm{\X \u}{2}^2 $ is known as Restricted Eigenvalue (RE) condition. 
	The upper bound factorizes as:
	\be 
	\label{eq:tub}
	\frac{2}{n}\w^T \X\ddelta &\leq& \frac{2}{n} \sup_{\u \in \bcH} \w^T \X \u \left(\sum_{g=0}^{G} \sqrt{\frac{n_g}{n}} \norm{\ddelta_g}{2} \right) , \quad \u \in \cH \\ \nr 
	\ee 
	Putting together inequalities \eqref{eq:tre} and \eqref{eq:tub} completes the proof. 
\end{proof}

\begin{remark}
Consider the setting where $n_g = \Theta(\frac{n}{G})$ so that each group has approximately $\frac{1}{G}$ fraction of the samples. Then, $\gamma = \Theta(G)$ and hence
\beq
\nr 
\frac{1}{G} \sum_{g=0}^G \| \delta_g \|_2 \leq O( G^{1/2} ) \frac{\sup_{\u \in \bcH}\oomega^T \X \u}{n}~.
\eeq
\end{remark}

%We define $\gamma_g = \norm{\ddelta_g}{2}$ and for $\ddelta \in \cH$ we have $\sum_{g = 0}^{G} \gamma_g = 1$.
%We are interested in the following RE condition:
%\begin{remark}
%	As we show in the following, the lower bound $\kappa  < \inf_{\u \in \cC} \norm{\X \u}{2}^2 $, holds even for the larger set $\cC \supset \cA$, which is an interesting result.	On the other hand, if we work with the set $\cC$ in the upper bound, the bound becomes loose.
%	Therefor, we focus on the set $\cA$ to get the tighter upper bound.
%\end{remark}
