\begin{figure}
	\centering
	\begin{subfigure}[b]{0.22\textwidth}
		\includegraphics[width=\textwidth]{./img/dr4.pdf}
		\caption{}\label{fig:dr}
	\end{subfigure} ~
	\begin{subfigure}[b]{0.22\textwidth}
		\includegraphics[height=2.7cm]{./img/Saracatinib.pdf}
		\caption{}\label{fig:Saracatinib}
	\end{subfigure}
	\squeezeup
	\caption{a) A sample fitted dose-response curve where Activity Area $y_{gi}$ is shaded. b) Distribution of responses to Saracatinib for which some lung and blood cancer cell lines have responded.}
	\label{fig syn3}
\end{figure}
\section{Anti-Cancer Drug Sensitivity Prediction}
\label{realexp}
In this section, we investigate the application of \dc{} in analyzing the response of cancer tumor cell lines to different doses of various drugs. 
Each cancer type (lung, blood, etc.) is a group $g$ in our DE model and the respond of patient $i$ with cancer $g$ to the drug is our output $y_{gi}$. 
The set of features for each patient $\x_{gi}$ consists of gene expressions, copy number variation, and mutations and $y_{gi}$ is the ``activity area'' above the dose-response curve, Figure \ref{fig:dr}.
Given $\x_{gi}$ and a drug, we want accurately predict a patient's response to a drug and identifying genetic predictors of drug sensitivity. 
%We present more details about and results of the drug sensitivity experiment in Appendix \ref{?}.
%In other words, we are interested in both predicting of $y_{gi}$ and identifying feature $j$ with highest absolute $\bbeta_{g}(j)$ value for each cancer $g$.
%We hope that the data enriched model captures the important shared feature between cancers, as long as the cancer-specific genetic factors that determine the response to therapy.
%In the following, we first explain the dataset and then compare the prediction performance of DS with elastic net on 24 drugs. 
%Then, we perform a more detailed study of the features selected by DS to validate if DS extracts meaningful feature for each cancer. 
%\subsection{Data Set}
%A dose-response dataset of an anticancer drug $d$ and a cell line $gi$ (sampled from a patient $i$ with cancer $g$) consists of pairs ${(c, p)}$ which means proportion $p$ of cells of the cell line $gi$ are alive (relative viability) when treated with drug $d$ with concentration $c$. 
%%There are many methods that fit different function to dose-response data, 
%When a drug \emph{works}, after fitting a function to raw data, we will have a reverse sigmoid-like graph, e.g., Figure \ref{fig:dr}. 
%An effective drug kills more cells with lower concentrations therefore its  ``activity area'' \cite{zoha05}, i.e., area above the curve in Figure \ref{fig:dr}, is larger.
%%There are many ways to measure the drug efficacy from the dose-response curve. 
%We use the activity area of the drug for cell line $gi$ as our output of interest $y_{gi}$. 
%
%\begin{figure}
%	\centering
%	\includegraphics[width=0.3 \textwidth,]{./img/dr4}
%	\caption{}
%	\label{fig:dr}
%\end{figure}
%\vspace{-3mm}
\begin{table*}[bp]
	\centering
	{\small
		\begin{tabular}{ |c|c|c|c|  }
			\hline 	
			\multicolumn{2}{|c|}{Blood and Lymph} & \multicolumn{2}{c|}{Lung}\\
			\hline
			Highlights &    p-Val  & Highlights &    p-Val   \\
			\hline
			Viral leukemogenesis		 & 6.17E-4  & Primary mucoepidermoid carcinoma of lung & 1.85E-4  \\
			Primary cutaneous marginal zone B-cell lymphoma	 & 1.85E-3   & Lung carcinoma cell type unspecified stage IV	& 1.85E-4   \\
			Burkitt Lymphoma	 & 5.50E-3  & Primary adenocarcinoma of lung  & 1.85E-4 \\
			\hline
		\end{tabular}
		\caption{Highlights of interpretable outcomes of \dc{}. p-Values are computed by  Fisher's exact test \cite{chen09toppgene}.}
		\label{table:1}}
\end{table*}
%In this experiment 
We use Cancer Cell Line Encyclopedia (CCLE) \cite{barretina2012cancer} which is a compilation $\sim$500 human cancer cell lines for 36 cancer types where their responses to 24 anticancer drugs have been measured.%\footnote{Here the technical meaning of response $y_{gi}$ is the ``activity area'' above the dose-response curve\cite{barretina2012cancer}.}.
%The goal of the study was to identify genetic predictor of drug sensitivity for different cancers. 
%The method used for analysis of CCLE in the original paper was Elastic Net \cite{zoha05} which we use as the comparison baseline. 
%\footnote{Cancer type is the first tissue where the cancer was observed.} 
We perform two \emph{experiments} where the number of cancers in each data set are $G =2$ or $3$ and we name them TWO and THREE experiments, respectively. 
We consider lung and blood\footnote{By blood cancer, we mean any cancer originate from haematopoietic and lymphoid tissues.} 
for TWO while for THREE we predict the drug sensitivity of skin, breast and ovary cancer cell lines. 
Beyond these five cancer types, others have less than 50 samples, so we remove them from consideration.
Each experiment consists of 24 \emph{problems} each corresponds to a drug. % drug sensitivity prediction of 
Not all of the $~$500 cell lines have been treated with all of the drugs. 
Therefore each problem has a different number of samples $n$ where $n \in [90, 160]$ for TWO and $n \in [70, 100]$ for THREE experiments.
We perform a standard preprocessing \cite{barretina2012cancer} where we remove features with less than $.2$ absolute correlation with the response.% of interest.% which reduces the dimension from $>30,000$ to $p \in [?, ?]$.% range. 
The features that get removed vary by problem, so the dimension $p$ is reduced from from $>30,000$ to $p \in [1000, 15000]$. 
	\begin{figure}
	\centering
	\begin{subfigure}[b]{0.22\textwidth}
		\includegraphics[width=\textwidth]{./img/lung-blood-barplot.pdf}
		\caption{Lung and Blood}\label{fig:two}
	\end{subfigure} ~
	\begin{subfigure}[b]{0.22\textwidth}
		\includegraphics[width=\textwidth]{./img/skin-ovary-breast.pdf}
		\caption{Skin, Ovary, and Breast}\label{fig:three}
	\end{subfigure}
	\squeezeup
	\caption{Distribution of MSE Comparison of Mean Square Error of \dc{} and BL in predicting the response to 24 drugs in TWO and THREE experiments. Each bar is the mean of MSE for 5-fold cross-validation.}
	\label{fig syn4}
\end{figure}
%\subsection{Results}

{\bf Prediction:} In each TWO and THREE experiments, we predict the drug sensitivity for 24 different drugs using sparse DE estimator \eqref{sde}.  %= \{f_g(\bbeta) < b_g\}$
Since the values of $d_g$  in constraint sets $\Omega_{f_g}(d_g)$  are unknown, we tune them by 5-fold cross-validation and report the mean squared error (MSE) of \dc{} and a baseline method. 
Our \emph{baseline} method BL is the LASSO \cite{tibs96} equivalent of DE where we set $\forall g \in [G]_\setminus d_g = 0$ and only estimate the common parameter $\bbeta_0$. 
Figure \ref{fig:two} and \ref{fig:three} illustrate the performance of \dc{} and BL for both experiments. 
Note that \dc{} outperforms BL in 21 and 18 out of 24 problems in TWO and THREE experiments, respectively.

To ensure that the prediction improvement of \dc{} over the baseline is statistically significant, we supplement our analysis with the bootstrapped error of both methods for the TWO experiment.
For each problem in the TWO experiment, we generate 100 bootstrapped data sets by sampling with replacement as $\{(\X^{(i)}_{\text{TWO}}, \y^{(i)}_{\text{TWO}})\}_{i=1}^{100}$.  
Then, we fix $d_g$s hyper-parameters to values determined by cross-validation in the last stage and run both methods and compute pairs of MSEs as $\{(\text{MSE}^{(i)}_{\text{\dc}}, \text{MSE}^{(i)}_{\text{BL}})\}_{i=1}^{100}$ for each problem (drug). We perform paired t-test to determine if difference between means of two methods' MSEs is significant. In 21 out of 24 problems \dc's MSE is lesser than BL's with significance level of $\alpha = 0.05$. A representative set of results is demonstrated in Figure \ref{fig:allbars}. 

	




	\begin{figure}
	\centering
	\begin{subfigure}[b]{0.22\textwidth}
		\centering
		\includegraphics[width=\textwidth]{./img/bar1.pdf}
		\caption{}    
%		\label{fig:mean and std of net14}
	\end{subfigure}
	\hfill
	\begin{subfigure}[b]{0.22\textwidth}  
		\centering 
		\includegraphics[width=\textwidth]{./img/bar2.pdf}
		\caption{}    
%		\label{fig:mean and std of net14}
	\end{subfigure}
	\vskip\baselineskip
	\begin{subfigure}[b]{0.22\textwidth}   
		\centering 
		\includegraphics[width=\textwidth]{./img/bar3.pdf}
		\caption{}    
%		\label{fig:mean and std of net14}
	\end{subfigure}
	\quad
	\begin{subfigure}[b]{0.22\textwidth}   
		\centering 
		\includegraphics[width=\textwidth]{./img/bar4}
		\caption{}    
%		\label{fig:mean and std of net14}
	\end{subfigure}
	\squeezeup
	\caption{MSE for 100 bootstrapped dataset of four drugs of experiment TWO. (a),(b) Sample result of large difference between mean of MSEs and small p-values. (c) Smaller mean difference with significant p-value. (d) One of the three cases where \dc{} is outperformed by the baseline.} 
	\label{fig:allbars}
	\end{figure}


{\bf Interpretation}
%Next we focus on the interpretation of the selected genes. 
We select Saracatinib, a drug which shows activity on both lung and blood cancer cell lines, Figure \ref{fig:Saracatinib}. 
Then, during the bootstrap experiment on TWO, we record support of the estimated parameters by \dc. 
%Fixing the $d_g$ parameters, 
We pick the top five most frequently selected genes across 100 bootstrapped runs for further analysis. 
Now, we have three lists of genes for common, lung , and blood parameters. 
We perform gene enrichment analysis using ToppGene \cite{chen09toppgene} to see where in functional/disease/drug databases these genes have been observed together with statistical significance. 
Table \ref{table:1} summarizes a highlight of our findings which shows lung and blood parameters' supports are capturing a meaningful set of genes as a biomarkers.% for anticancer drug sensitivity.% while the shared parameter has mixed set of genes. 
%{\small 
%\begin{tabular}{ |c|c|c|c|c|c|  }
%	\hline 	
%	\multicolumn{2}{|c|}{(Blood, 512)} & \multicolumn{2}{c|}{(Lung, 500)} & \multicolumn{2}{c|}{(Shared, 525)}\\
%	\hline
%	Highlights &    p-Val  & Highlights &    p-Val &  Highlights &    p-Val \\
%	\hline
%	regulation of immune response	 & 2.1E-8  & Secondary malignant neoplasm of lung & 8.9E-6  &  Acute	Myeloid Leukemia & 5.0E-7 \\
%	T cell activation	 & 5.0E-8   & Lung Cancer  & 2.9E-5  &  Chronic Myeloid Leukemia & 3.0E-5 \\
%	leukocyte activation & 1.0E-6  & Adenosquamous cell lung cancer	 & 3.9E-5  &  Adenocarcinoma of lung & 4.8E-5 \\
%	\hline
%\end{tabular}
%}

%{ (Type, $|\text{supp}(\cdot)|$)} 	&   , ; 2.907E-5  & asd;flkajsdf ;as  &  asdf'alsdkfad' \\
%&   , stage IV;   & asd;flkajsdf ;as  &  asdf'alsdkfad' \\