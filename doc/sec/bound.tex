\section{General Error Bound}
\label{sec:error}
In this section, we provide a high probability upper bound for the estimation error of the common and individual parameters.% with the structure inducing convex functions $f_g(\cdot)$s.
To avoid cluttering the notation, we rename the vector of all noises as $\oomega_0 \triangleq \oomega$.%\ab{We are using $\omega$ to denote two different things---the Gaussian width and the noise!}
First, we massage the deterministic upper bound of Theorem \ref{theo:deter} as follows:
	\be
	\nr
	\oomega ^T \X\ddelta &=& \sum_{g=0}^{G} \langle \X_g^T \oomega _g,  \ddelta_g \rangle
	\\ \nr
	&=& \sum_{g=0}^{G} \sqrt{\frac{n_g}{n}} \norm{\ddelta_g}{2} \langle \X_g^T \frac{\oomega _g}{\norm{\oomega _g}{2}}, \frac{\ddelta_g}{\norm{\ddelta_g}{2}} \rangle \sqrt{\frac{n}{n_g}} \norm{\oomega _g}{2} \\ \nr
	%(\forall g: \u_g \in \cC_g \cap \sphere) &=& \norm{\ddelta_0}{2} \langle \X_0^T \frac{\oomega }{\norm{\oomega }{2}}, \u_0 \rangle \norm{\oomega }{2} + \sum_{g=1}^{G} \norm{\ddelta_g}{2} \langle \X_g^T \frac{\oomega _g}{\norm{\oomega _g}{2}}, \u_g \rangle \norm{\oomega _g}{2} \\ \nr
	\ee
Assume $b_g = \langle \X_g^T \frac{\oomega _g}{\norm{\oomega _g}{2}}, \frac{\ddelta_g}{\norm{\ddelta_g}{2}}  \rangle \sqrt{\frac{n}{n_g}} \norm{\oomega _g}{2}$ and $a_g = \sqrt{\frac{n_g}{n}} \norm{\ddelta_g}{2}$.
Then the above term is the inner product of two vectors $\a = (a_0, \dots, a_G)$ and $\b = (b_0, \dots, b_G)$ for which we have:
%\be
%\nr
%%\sup_{\a \in \cH_l} \a^T \b  &\leq& \sup_{\a \in \cH_0} \a^T \b
%%\\ \nr
%\sup_{\a \in \cH} \a^T \b
%&=&\sup_{\norm{\a}{1} = 1} \a^T \b \\ \nr
%\text{(definition of the dual norm)} &\leq& \norm{\b}{\infty} \\ \nr
%&=& \max_{g \in [G]} b_g \nr
%\ee
\be
\nr
%\sup_{\a \in \cH_l} \a^T \b  &\leq& \sup_{\a \in \cH_0} \a^T \b
%\\ \nr
\sup_{\a \in \cH} \a^T \b
=\sup_{\norm{\a}{1} = 1} \a^T \b
&\leq& \norm{\b}{\infty}
= \max_{g \in [G]} b_g,
\ee
where the inequality holds because of the definition of the dual norm.
We can upper bounds $b_g$s with high probability and then by union bound below theorem establishes a high probability upper bound for the deterministic bound of Theorem \ref{theo:deter}, i.e., $\frac{2}{n}\oomega ^T \X \u$. %, in terms of the Gaussian width of the spherical caps corresponding to each error cone, i.e., $\omega(\cC_g \cap \sphere)$.

\begin{theorem}
	\label{theo:ub}
	Assume $\x_{gi}$ and $\omega_{gi}$ distributed according to Definition \ref{def:obs}, then with probability at least  $1 - \sigma \exp\left(-\min_{g \in [G]}\left[\nu_g  n_g - \log (G+1), \frac{\tau^2}{\eta_g^2 k^2}\right]\right) $  we have:
%	Assume $\x_{gi}$ to be a sub-Gaussian random variable with $\ex [\x_{gi}^T \x_{gi}] = \I_{p \times p}$ and $\vertiii{\x_{gi}}_{\psi_2} \leq k$ and $\oomega $ consists of i.i.d. centered unit-variance sub-Gaussian elements with $\normth{\omega_{gi}}{\psi_2} \leq K$, with probability at least  $1 - \sigma \exp\left(-\min_{g \in [G]}\left[\nu_g  n_g - \log (G+1), \frac{\tau^2}{\eta_g^2 k^2}\right]\right) $  we have:
	{\small
	\be
	\nr
	\frac{2}{n} \oomega ^T \X\ddelta
	&\leq& \sqrt{\frac{8 K^2 + 4}{n}} \max_{g \in [G]} \left(\zeta_g k \omega(\cA_g) + \epsilon_g \sqrt{\log (G+1)}+ \tau \right) \nr
	\ee
	}
	where $\sigma = \max_{g \in [G]} \sigma_g$ and $\tau > 0$.
\end{theorem}

The following corollary characterizes the general error bound and results from the direct combination of Theorem \ref{theo:deter}, Theorem \ref{theo:re}, and Theorem \ref{theo:ub}.
\begin{corollary}
	\label{corr:calcub}
	For $\x_{gi}$ and $\omega_{gi}$ described in Definition \ref{def:obs} when we have enough number of samples $\forall g \in [G]: n_g > m_g$ which lead to $\kappa > 0$, the following general error bound holds with high probability for estimator \eqref{eq:compact}:
%	at least{\small  $1 - \sigma \exp\left(-\min_{g \in [G]} \left[\nu_g  n_g - \log (G+1), \frac{\tau^2}{\eta_g^2 k^2}\right]\right)$}
	{\small\be
	\label{eq:general}
	\sum_{g=0}^{G} \sqrt{\frac{n_g}{n}} \norm{\ddelta_g}{2}
	\leq C {\gamma} \frac{k\zeta \max_{g \in [G]}  \omega(\cA_g) + \epsilon \sqrt{\log (G+1)}+ \tau }{\kappa_{\min}^2 \sqrt{n}}
	\ee	}
	where $C = 8\sqrt{2 K^2 + 1}$, $\zeta = \max_{g \in [G]} \zeta_g$, $\epsilon = \max_{g \in [G]} \epsilon_g$, $\gamma = \max_{g \in [G]_{\setminus}} n/n_g$  and $\tau > 0$.
\end{corollary}

\ab{This section will be very hard to follow without examples. Lets use the two examples we (will) introduce earlier.}

%\begin{corollary}
%	\label{corr:single}
%	Note that from \eqref{eq:general} one can immediately entail the error bound for estimation of the common and individual parameters as follows:
%	\be
%	\nr
%	%\forall g \in [G]:
%%	\forall g \in [G]: \quad \norm{\ddelta_g}{2} \leq \sqrt{\gamma} \sqrt{\frac{n}{n_g}} O\left(\frac{\max_{g \in [G]}  \omega(\cA_g) + \sqrt{\log (G+1)} }{\sqrt{n_g}}\right)
%	\forall g \in [G]: \quad \norm{\ddelta_g}{2} \leq {\gamma} O\left(\frac{\max_{g \in [G]}  \omega(\cA_g) + \sqrt{\log (G+1)} }{\sqrt{n_g}}\right)
%	\ee
%\end{corollary}

%	{\color{red} Do we need these remarks anymore? We have the extra factor
%		
%	Comparing the result of Corollary \ref{corr:single} with the case of regression with the single structured parameter $\bbeta_g^*$ is instructive.
%	Following are some remarks comparing our results with the state-of-the-art.
%\begin{remark}
%	Corollary \ref{corr:single} states $\forall \in [G]_\setminus: \norm{\ddelta_g}{2} \leq O((\max_{g \in [G]} \omega(\cA_g) + \sqrt{\log (G+1)})/\sqrt{n_g})$ while sharp error bound for the single regression with $\bbeta_g^*$ is $\norm{\ddelta_g}{2} \leq O(\omega(\cA_g)/\sqrt{n_g})$.
%	So by solving a more complicated data enriched model we only pay a price of $\big(\max_{g \in [G]} \omega(\cA_g) - \omega(\cA_g) + \sqrt{\log (G+1)}\big)/ \sqrt{n_g}$ in estimation error while the order of the sample complexity $m_g$ stays the same.
%\end{remark}
%
%\begin{remark}
%	On the other hand, without any direct observation regarding the parameter $\bbeta _0^*$ we exploit all of the groups data and get the decay rate of $1/\sqrt{n}$ for $\norm{\ddelta_0}{2}$ by only paying a price of $\big(\max_{g \in [G]} \omega(\cA_g) - \omega(\cA_0)+ \sqrt{\log (G+1)}\big)/ \sqrt{n}$ in estimation error.
%\end{remark}
%}
%\begin{remark}
%	\label{rem:sperror}
%	For sparse parameters, assume that each $\bbeta _g^*$ is $s_g$-sparse and $s = \max_{g \in [G]} s_g$, i.e., the densest parameter is $s$-sparse.
%	Then we have the following error bounds with high probability when number of per group samples $n_g = O(s_g \log p)$ and total number of samples $n = O(s_0 \log p)$ :
%	\be
%	\nr
%%	\forall g \in [G]: \norm{\ddelta_g}{2} \leq c \sqrt{\gamma} \sqrt{\frac{n}{n_g}} \frac{\sqrt{s  \log p} + \sqrt{\log (G+1)}}{\sqrt{n_g}}
%	\forall g \in [G]: \norm{\ddelta_g}{2} \leq c {\gamma} \frac{\sqrt{s  \log p} + \sqrt{\log (G+1)}}{\sqrt{n_g}}
%	\ee
%%	Note that here the recovery of the common parameter is at most $c\sqrt{\frac{\sqrt{\log G}}{n}}$ worse than the case of single regression with $\bbeta _0$ as the parameter.
%%	Also for the individual parameters, the bound is only $c \frac{(\sqrt{s_0} - \sqrt{s_g}) \sqrt{\log p} + \sqrt{\sqrt{\log G}}}{\sqrt{n_g}}$ weaker than the case of single regression.
%\end{remark}

%\section{Minimax Lower Bound}

